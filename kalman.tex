\documentclass[12 pt]{article}
\usepackage[utf8]{inputenc}
\usepackage[utf8]{inputenc}
\usepackage[T1]{fontenc}
%\usepackage{fullpage}
\usepackage{graphicx}
\usepackage{mathtools}
\usepackage{amsmath}
\usepackage{hyperref}
\usepackage{gensymb}
\usepackage[brazilian]{babel}

\begin{document}

\begin{titlepage}

\begin{center}
\textbf{\Large Ministério da Defesa}\\\
\textbf{\Large Exército Brasileiro}\\
\textbf{\Large Secretaria de Ciência e Educação}\\
\textbf{\Large CETEX \& IME }\\\vfill
\textbf{\huge Modelo do filtro de kalman para o futebol SSL}\\[2cm]%\vfill
\end{center}
\begin{center}
\textbf{\Large 1 Ten Bramigk \& AspOf R2 Castelo Branco}\\\vfill

\textbf{\large Rio de Janeiro}\\
\textbf{\large 2016}
\end{center}
\end{titlepage}

\section{Modelo}

A primeira equação do estado que temos é:

\begin{center}
$\hat{x}_{k} = A \hat{x}_{k-1} + B u_{k}$
\end{center}

Na qual $\hat{x}_{k} = (\hat{r}_{x}, \hat{r}_{y}, \hat{r}_{\theta})^{T}_{k}$ é  estado real medido no tempo k,  $\hat{x}_{k-1} = (\hat{r}_{x}, \hat{r}_{y}, \hat{r}_{\theta})^{T}_{k-1}$ é o estado real medido no tempo k-1. $A = I_{3}$ e $B = \Delta tR_{z}(\theta)$ são matrizes apropriadas e ${u}_{k} = (v_{t}, v_{n}, v_{\omega})^{T}$ é o deslocamento somado a cada tempo $\Delta t_{k} - \Delta t_{k-1}$. Ou seja:
\begin{center}
$\begin{bmatrix}
  \hat{r}_{x}\\        
  \hat{r}_{y}\\[0.3em]
  \hat{r}_{\theta} 
 \end{bmatrix}_{k}
 = I_{3}\begin{bmatrix}
  \hat{r}_{x}\\        
  \hat{r}_{y}\\[0.3em]
  \hat{r}_{\theta} 
 \end{bmatrix}_{k-1}
 + R_{z}(\theta)\begin{bmatrix}
  \Delta t v_{t}\\        
  \Delta t v_{n}\\[0.3em]
  \Delta t v_{\omega} 
 \end{bmatrix}$
\end{center}

A segunda é:
\begin{center}
$z_{k} = C x_{k} + v_{k}$
\end{center}

Na qual $z_{k}$ é o estado medido, $C = I_{3}$ é uma constante, $x_{k}$ é o estado real em k e $v_{k}$ é o ruído da medição.

A predição envolve também a fórmula $P_{k}=AP_{k-1}A^{T}$. Porém, $A = I_{3}$, então: $P_{k}=P_{k-1}$.

A correção é dada pelas fórmulas recursivas: 
\begin{center}
$G_{k} = P_{k}C^{T}(CP{k}C^{T}+R)^{-1} $\\
$\hat{x}_{k} = \hat{x}_{k} + G_{k}(z_{k} - C\hat{x}_{k})$\\
$P_{k} = (I_{3} - G_{k}C)P_{k-1}$
\end{center}

Como $A=I_{3}$ e $C=I_{3}$, e já pensando em rearranjá-las, temos que as fórmulas ficam:
\begin{center}
$\hat{x}'_{k} = \hat{x}_{k-1} + Bu_{k}$ (1)\\
$P_{k} = P_{k-1}$ (2)\\
$G_{k} = P_{k}(P{k}+R)^{-1}$ (3)\\
$\hat{x}_{k} = \hat{x}'_{k} + G_{k}(z_{k} - \hat{x}'_{k})$ (4)\\
\end{center}

Substituindo as equações (1), (2) e (3) na (4) temos que:
\begin{center}
$\hat{x}_{k} = \hat{x}_{k-1} + Bu_{k} + P_{k}(P{k}+R)^{-1}(z_{k} - (\hat{x}_{k-1} + Bu_{k}))$
\end{center}

Que é:
\begin{center}
$\hat{x}_{k} = \hat{x}_{k-1} + Bu_{k} + P_{k}(P{k}+R)^{-1}(z_{k} - \hat{x}_{k-1} - Bu_{k})$
\end{center}

Multiplicando por $P_{k}^{-1}$ pela esquerda temos que:

\begin{center}
$P_{k}^{-1}\hat{x}_{k} = P_{k}^{-1}(\hat{x}_{k-1} + Bu_{k}) + (P{k}+R)^{-1}(z_{k} - \hat{x}_{k-1} - Bu_{k})$
\end{center}

Supondo $P_{k} = pI_{3}$:

\begin{center}
$\frac{1}{p}\hat{x}_{k} = \frac{1}{p}(\hat{x}_{k-1} + Bu_{k}) + (pI_{3}+R)^{-1}(z_{k} - \hat{x}_{k-1} - Bu_{k})$
\end{center}

Multiplicando por $(pI_{3}+R)$ pela esquerda, temos que:

\begin{center}
$\frac{1}{p}(pI_{3}+R)\hat{x}_{k} = \frac{1}{p}(pI_{3}+R)(\hat{x}_{k-1} + Bu_{k}) + z_{k} - \hat{x}_{k-1} - Bu_{k}$
\end{center}

\begin{center}
$(I_{3}+R/p)\hat{x}_{k} = (I_{3}+R/p)(\hat{x}_{k-1} + Bu_{k}) + z_{k} - \hat{x}_{k-1} - Bu_{k}$
\end{center}

\begin{center}
$\hat{x}_{k} + \frac{R}{p}\hat{x}_{k} = \hat{x}_{k-1} + Bu_{k} +\frac{R}{p}\hat{x}_{k-1} + \frac{R}{p}Bu_{k} + z_{k} - \hat{x}_{k-1} - Bu_{k}$
\end{center}

Fazendo $R' = \frac{R}{p}$, temos:

\begin{center}
$\hat{x}_{k} + R'\hat{x}_{k} = R'\hat{x}_{k-1} + R'Bu_{k} + z_{k}$
\end{center}


\begin{center}
$R'(\hat{x}_{k} - \hat{x}_{k-1} - Bu_{k})= z_{k} - \hat{x}_{k}$
\end{center}

Agora temos que descobrir $R'$, usando três valores para $u_{k}$:

\begin{center}
$u_{k} = \begin{bmatrix}
       1           \\[0.3em]
       0\\[0.3em]
       0 
     \end{bmatrix}$, 
$u_{k} = \begin{bmatrix}
       0           \\[0.3em]
       1\\[0.3em]
       0 
     \end{bmatrix}$ e
$u_{k} = \begin{bmatrix}
       0           \\[0.3em]
       0\\[0.3em]
       1 
     \end{bmatrix}$
\end{center}


%$M = \begin{bmatrix}
%       \frac{5}{6}           \\[0.3em]
%       \frac{5}{6}\\[0.3em]
%       0 
%     \end{bmatrix}$


\end{document}

